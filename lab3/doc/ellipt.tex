\documentclass[12pt]{article}
\usepackage[utf8]{inputenc}
\usepackage[T2A]{fontenc}
\usepackage[english,russian]{babel}
\usepackage{amsmath,amssymb}

\author{Цыбулин Иван}
\title{Построение схем с положительной аппроксимацией для эллиптических уравнений безсеточным методом}

\renewcommand{\vec}[1]{\boldsymbol{\mathbf{#1}}}

\newcommand{\pd}[2]{\frac{\partial{#1}}{\partial {#2}}}
\newcommand{\td}[2]{\frac{d{#1}}{d {#2}}}

\newcommand{\cutefrac}[2]{{}^{#1\!}\!/{\!}_#2}
\newcommand{\half}{\cutefrac{1}{2}}

\begin{document}
\maketitle

\section{Постановка задачи}

Рассмотрим общий вид эллиптической задачи с постоянными коэффициентами ($e_{11} > 0, e_{22} > 0, 4e_{11}e_{22} > e_{12}^2$)
\begin{equation}
\begin{cases}
\mathcal{L} u \equiv
e_{11} u_{xx} + e_{12} u_{xy} + e_{22} u_{yy}
+ e_1 u_x + e_2 u_y + e_0 u = f(x, y), &\vec r \in G\\
a(x,y) u + b(x,y) \pd{u}{n} = g(x,y), &\vec r \in \partial G.\\
\end{cases}
\label{eq:problem}
\end{equation}

Схемой с положительной аппроксимацией для \eqref{eq:problem} назовем такую схему
\[
\sum_{\mu \in N(m)} \alpha_{\mu}^{(m)} u_{m+\mu} = \beta^{(m)},
\]
где $N(m)$ --- некоторый набор соседей для точки $m$, $m+\mu$ --- $\mu$-й сосед точки $m$. Обозначение $(m)$ показывает, что коэффициент относится к уравнению для точки $m$. 

Заметим, что для параболической задачи
\[
\pd{u}{t} = \mathcal{L} u - f
\]
простейшая схема может иметь вид
\[
\frac{\hat u_m - u_m}{\tau} = 
\sum_{\mu \in N(m)} \alpha_{\mu}^{(m)} u_{m+\mu} - \beta^{(m)}
\]
и использоваться как метод установления для решения исходной эллиптической задачи.

Условием монотонности данной схемы установления будет 
\begin{gather*}
\alpha_{\mu}^{(m)} \geqslant 0, \quad \mu \neq 0\\
1 + \tau \alpha_0^{(m)} \geqslant 0 \Leftrightarrow \alpha_0^{(m)} \geqslant -\frac{1}{\tau}.
\end{gather*}
Последнее уравнение, фактически, задает ограничение на шаг по времени $\tau$:
\[
\tau \leqslant \frac{1}{-\operatorname{absmax}_m \alpha_0^{(m)}}.
\]

\subsection{Аппроксимация дифференциального оператора}
Пусть точка $m$ окружена множеством $N(m)$ соседей, причем радиус вектор из точки $m$ в точку $m + \mu$ обозначим как $\vec r_\mu^{(m)}$. Далее будем отпускать верхний индекс $(m)$, подразумевая некую фиксированную точку.
\[u_{m+\mu} = \exp\{\vec r_{\mu}^{(m)}\nabla\} u_m = 
\sum_{k=0}^\infty \frac{\left(x_\mu\pd{}{x} + y_\mu\pd{}{y}\right)^k}{k!} u_m
\]
Чтобы аппроксимировать $\mathcal{L}u$ с минимальным порядком $O(h)$, напишем разложения с точностью до $O(h^3)$ (учитываем, что $\alpha = O(h^{-2})$).
\[
u_{m+\mu} = u_m +
x_\mu (u_x)_m + y_\mu (u_y)_m + 
\frac{x_\mu^2}{2} (u_{xx})_m +
x_\mu y_\mu (u_{xy})_m +
\frac{y_\mu^2}{2} (u_{yy})_m + O(h^3).
\]
Из условия аппроксимации
\[
\sum \alpha_\mu u_{m+\mu} - \beta_m = \mathcal{L} u_m - f_m + O(h)
\]
можно получить систему уравнений на $\alpha_\mu$:
\[
\begin{aligned}
u: &\sum_\mu \alpha_\mu = e_0\\
u_x: &\sum_\mu \alpha_\mu x_\mu = e_1\\
u_y: &\sum_\mu \alpha_\mu y_\mu = e_2
\end{aligned}
\qquad\qquad
\begin{aligned}
u_{xx}: &\sum_\mu \alpha_\mu \frac{x_\mu^2}{2} = e_{11}\\
u_{xy}: &\sum_\mu \alpha_\mu x_\mu y_\mu = e_{12}\\
u_{yy}: &\sum_\mu \alpha_\mu \frac{y_\mu^2}{2} = e_{22}.
\end{aligned}
\]

При общем положении необходимо $5$ соседей (плюс сама точка, итого $6$ коэффициентов $\alpha$), чтобы разрешить 6 уравнений относительно $\alpha_\mu$. На шаблоне <<крест>> достаточно четырех соседей. Приравнивая правые части, получаем условие для $\beta_m$
\[
\beta_m = f_m + O(h).
\]

Чтобы повысить порядок аппроксимации хотя бы до $O(h^2)$ \emph{на решениях уравнения} можно воспользоваться следствиями дифференциального уравнения
\[
\mathcal{L} u_x = f_x, \qquad \mathcal{L} u_y = f_y.
\]
Необходимо построить аппроксимацию
\begin{multline*}
\sum \alpha_\mu u_{m+\mu} - \beta_m = \mathcal{L} u_m - f_m + \\
+ \gamma_1 \left[\mathcal{L} (u_x)_m - (f_x)_m\right]
+ \gamma_2 \left[\mathcal{L} (u_y)_m - (f_y)_m\right]
+ O(h^2)
\end{multline*}
где $\gamma_1, \gamma_2$ --- произвольные коэффициенты порядка $O(h)$.
\[
\begin{aligned}
u: &\sum_\mu \alpha_\mu = e_0\\
u_x: &\sum_\mu \alpha_\mu x_\mu = e_1 + \gamma_1 e_0\\
u_y: &\sum_\mu \alpha_\mu y_\mu = e_2 + \gamma_2 e_0\\
u_{xx}: &\sum_\mu \alpha_\mu \frac{x_\mu^2}{2} = e_{11} + \gamma_1 e_1\\
u_{xy}: &\sum_\mu \alpha_\mu x_\mu y_\mu = e_{12} + \gamma_1 e_2 + \gamma_2 e_1\\
u_{yy}: &\sum_\mu \alpha_\mu \frac{y_\mu^2}{2} = e_{22} + \gamma_2 e_2
\end{aligned}
\qquad
\begin{aligned}
u_{xxx}: &\sum_\mu \alpha_\mu \frac{x_\mu^3}{2} = \gamma_1 e_{11}\\
u_{xxy}: &\sum_\mu \alpha_\mu \frac{x_\mu^2 y_\mu}{2} = \gamma_1 e_{12} + \gamma_2 e_{22}\\
u_{xyy}: &\sum_\mu \alpha_\mu \frac{x_\mu y_\mu^2}{2} = \gamma_1 e_{22} + \gamma_2 e_{12}\\
u_{yyy}: &\sum_\mu \alpha_\mu \frac{y_\mu^3}{2} = \gamma_2 e_{22}.
\end{aligned}
\]
При этом для $\beta$ имеем
\[
\beta = f + \gamma_1 f_x + \gamma_2 f_y.
\]

Исключая из уравнений для $u_{xxx}, u_{yyy}$ величины $\gamma_{1,2}$, имеем систему из 8 уравнений
\begin{align*}
u: &\sum_\mu \alpha_\mu = e_0\\
u_x: &\sum_\mu \alpha_\mu x_\mu = e_1 + \frac{e_0}{e_{11}}\sum_\mu \alpha_\mu \frac{x_\mu^3}{6}\\
u_y: &\sum_\mu \alpha_\mu y_\mu = e_2 + \frac{e_0}{e_{22}}\sum_\mu \alpha_\mu \frac{y_\mu^3}{6}\\
u_{xx}: &\sum_\mu \alpha_\mu \frac{x_\mu^2}{2} = e_{11} + \frac{e_1}{e_{11}}\sum_\mu \alpha_\mu \frac{x_\mu^3}{6}\\
u_{xy}: &\sum_\mu \alpha_\mu x_\mu y_\mu = e_{12} + \frac{e_2}{e_{11}}\sum_\mu \alpha_\mu \frac{x_\mu^3}{6} + \frac{e_1}{e_{22}}\sum_\mu \alpha_\mu \frac{y_\mu^3}{6}\\
u_{yy}: &\sum_\mu \alpha_\mu \frac{y_\mu^2}{2} = e_{22} + \frac{e_2}{e_{22}}\sum_\mu \alpha_\mu \frac{y_\mu^3}{6}\\
u_{xxy}: &\sum_\mu \alpha_\mu \frac{x_\mu^2 y_\mu}{2} = \frac{e_{12}}{e_{11}}\sum_\mu \alpha_\mu \frac{x_\mu^3}{6} + \frac{e_{11}}{e_{22}}\sum_\mu \alpha_\mu \frac{y_\mu^3}{6}\\
u_{xyy}: &\sum_\mu \alpha_\mu \frac{x_\mu y_\mu^2}{2} = \frac{e_{22}}{e_{11}}\sum_\mu \alpha_\mu \frac{x_\mu^3}{6} + \frac{e_{12}}{e_{22}}\sum_\mu \alpha_\mu \frac{y_\mu^3}{6}.
\end{align*}

Запишем это в виде единой системы уравнений относительно $\alpha_\mu$:
\begin{align*}
u: &\sum_\mu \alpha_\mu = e_0\\
u_x: &\sum_\mu \alpha_\mu \left(x_\mu - \frac{e_0}{e_{11}}\frac{x_\mu^3}{6}\right) = e_1\\
u_y: &\sum_\mu \alpha_\mu \left(y_\mu - \frac{e_0}{e_{22}}\frac{y_\mu^3}{6}\right) = e_2\\
u_{xx}: &\sum_\mu \alpha_\mu \left(\frac{x_\mu^2}{2} - \frac{e_1}{e_{11}} \frac{x_\mu^3}{6} \right) = e_{11} \\
u_{xy}: &\sum_\mu \alpha_\mu \left(x_\mu y_\mu - \frac{e_2}{e_{11}} \frac{x_\mu^3}{6} - \frac{e_1}{e_{22}}\frac{y_\mu^3}{6}\right)= e_{12}\\
u_{yy}: &\sum_\mu \alpha_\mu \left(\frac{y_\mu^2}{2} - \frac{e_2}{e_{22}} \frac{y_\mu^3}{6}\right) = e_{22}\\
u_{xxy}: &\sum_\mu \alpha_\mu \left(\frac{x_\mu^2 y_\mu}{2} - \frac{e_{12}}{e_{11}}\frac{x_\mu^3}{6} - \frac{e_{11}}{e_{22}}\frac{y_\mu^3}{6}\right) = 0\\
u_{xyy}: &\sum_\mu \alpha_\mu \left(\frac{x_\mu y_\mu^2}{2} - \frac{e_{22}}{e_{11}}\frac{x_\mu^3}{6} - \frac{e_{12}}{e_{22}}\frac{y_\mu^3}{6}\right) = 0.
\end{align*}
Заметим, что во все уравнения кроме первого не входит $\alpha_0$. Ее можно исключить вместе с первым уравнением в системе
\[
\alpha_0 = e_0 - \sum_{\mu > 0} \alpha_\mu
\]
Для $\beta$
\[
\beta = f + \sum_\mu \alpha_\mu \left(\frac{x_\mu^3f_x}{6e_{11}} + \frac{y_\mu^3f_y}{6e_{22}}\right)
\]
Оставшаяся система (считаем, что $\mu \neq 0$)
\begin{align*}
	u_x: &\sum_\mu \alpha_\mu \left(x_\mu - \frac{e_0}{e_{11}}\frac{x_\mu^3}{6}\right) = e_1\\
	u_y: &\sum_\mu \alpha_\mu \left(y_\mu - \frac{e_0}{e_{22}}\frac{y_\mu^3}{6}\right) = e_2\\
	u_{xx}: &\sum_\mu \alpha_\mu \left(\frac{x_\mu^2}{2} - \frac{e_1}{e_{11}} \frac{x_\mu^3}{6} \right) = e_{11} \\
	u_{xy}: &\sum_\mu \alpha_\mu \left(x_\mu y_\mu - \frac{e_2}{e_{11}} \frac{x_\mu^3}{6} - \frac{e_1}{e_{22}}\frac{y_\mu^3}{6}\right)= e_{12}\\
	u_{yy}: &\sum_\mu \alpha_\mu \left(\frac{y_\mu^2}{2} - \frac{e_2}{e_{22}} \frac{y_\mu^3}{6}\right) = e_{22}\\
	u_{xxy}: &\sum_\mu \alpha_\mu \left(\frac{x_\mu^2 y_\mu}{2} - \frac{e_{12}}{e_{11}}\frac{x_\mu^3}{6} - \frac{e_{11}}{e_{22}}\frac{y_\mu^3}{6}\right) = 0\\
	u_{xyy}: &\sum_\mu \alpha_\mu \left(\frac{x_\mu y_\mu^2}{2} - \frac{e_{22}}{e_{11}}\frac{x_\mu^3}{6} - \frac{e_{12}}{e_{22}}\frac{y_\mu^3}{6}\right) = 0.
\end{align*}
должна иметь неотрицательное решение $\alpha_\mu \geqslant 0$. Обозначим систему
\[
\mathbf{M} \boldsymbol{\alpha} = \mathbf{b}
\]

С целью оптимизации количества ненулевых коэффициентов решается следующая задача линейного программирования
\begin{align*}
\underset{\alpha_\mu}{\operatorname{min}}\quad&\sum_\mu \alpha_\mu (x_\mu^2 + y_\mu^2)^{3/2}\\
\operatorname{s.t.}\quad&\mathbf{M} \boldsymbol{\alpha} = \mathbf{b}\\
&\boldsymbol{\alpha} \geqslant \mathbf{0}
\end{align*}
Данная задача не всегда может иметь решение, т.к. $\mathbf{M} \boldsymbol{\alpha} = \mathbf{b}$ может не иметь решения в неотрицательных $\alpha$. В этом случае необходимо привлекать дополнительные точки $\mu$.

\subsection{Аппроксимация граничных условий}

Пусть на границе области заданы граничные условия
\[
\omega\nu_x u_x + \omega\nu_y u_y + \sigma u = \rho,
\]
причем вектор $\boldsymbol{\nu}$ не обязан быть коллинеарен нормали $\mathbf{n}$, но должен выходить из области.

Необходимо построить аппроксимацию
\begin{multline*}
	\sum \alpha_\mu u_{m+\mu} - \beta_m = \rho_m - (\omega\nu_x u_x + \omega\nu_y u_y + \sigma u)_m 
	+ \gamma \left[\mathcal{L} u_m - f_m\right]
	+ O(h^2)
\end{multline*}
где $\gamma_1, \gamma_2$ --- произвольные коэффициенты порядка $O(h)$.
\[
\begin{aligned}
u: &\sum_\mu \alpha_\mu = -\sigma + \gamma e_0 \\
u_x: &\sum_\mu \alpha_\mu x_\mu = -\omega\nu_x + \gamma e_1\\
u_y: &\sum_\mu \alpha_\mu y_\mu = -\omega\nu_y + \gamma e_2
\end{aligned}
\qquad
\begin{aligned}
u_{xx}: &\sum_\mu \alpha_\mu \frac{x_\mu^2}{2} = \gamma e_{11}\\
u_{xy}: &\sum_\mu \alpha_\mu x_\mu y_\mu = \gamma e_{12}\\
u_{yy}: &\sum_\mu \alpha_\mu \frac{y_\mu^2}{2} = \gamma e_{22}
\end{aligned}
\]
При этом для $\beta$ имеем
\[
\beta = -\rho + \gamma f.
\]
Исключая $\gamma$ с помощью
\[
\gamma = \frac{1}{e_{11} + e_{22}} \sum_\mu \alpha_\mu\frac{x_\mu^2 + y_\mu^2}{2}
\]
\[
\begin{aligned}
u: &\sum_\mu \alpha_\mu = -\sigma + \gamma e_0 \\
u_x: &\sum_\mu \alpha_\mu \left(x_\mu - e_1\frac{x_\mu^2 + y_\mu^2}{2(e_{11} + e_{22})}\right) = -\omega\nu_x\\
u_y: &\sum_\mu \alpha_\mu \left(y_\mu - e_2\frac{x_\mu^2 + y_\mu^2}{2(e_{11} + e_{22})}\right) = -\omega\nu_y
\end{aligned}
\qquad
\begin{aligned}
u_{xy}: &\sum_\mu \alpha_\mu \left(
x_\mu y_\mu - e_{12}\frac{x_\mu^2 + y_\mu^2}{2(e_{11} + e_{22})}
\right)
 = 0\\
\Delta u: &\sum_\mu \alpha_\mu \left(e_{22} \frac{x_\mu^2}{2} - e_{11} \frac{y_\mu^2}{2}\right) = 0
\end{aligned}
\]
\end{document}